% -*- root: ../main.tex -*-

% Esporre l'obiettivo del progetto dandone una visione complessiva. Devono essere illustrate le caratteristiche salienti del progetto; deve essere chiara la distinzione tra le tecnologie usate/assemblate durante lo svolgimento dell'elaborato e il contributo tecnologico/scientifico e effettivamente apportato dal gruppo.

\chapter{Introduzione}


\section{Overview}
Weather Vortex è un progetto che nasce con l'intento di fornire all'utente un servizio per consultare e paragonare diverse previsioni meteo senza dover necessariamente consultare diversi portali o siti dedicati. L'obbiettivo verrà raggiunto mediante la creazione di una \textbf{piattaforma web} che possa, da un lato, \textbf{facilitare le decisioni dell'utente}, dall'altro \textbf{fornire un'esperienza personalizzata}.

\begin{figure}[H]
    \caption{Il logo della piattaforma}
    \label{fig:Logo}
    \centering
    \includegraphics[width=0.5\textwidth]{Images/logo.png}
\end{figure}

\section{Problemi}
Normalmente un utente consulta in media 3 / 4 siti meteo ogni volta che vuole farsi un'idea precisa delle previsioni del tempo. Questo comporta un dispendio di tempo e risorse nel consultare più fonti, quando un aggregatore potrebbe fare tutto in automatico.

\par Talvolta un utente appassionato potrebbe progettare una propria centralina meteo da collocare in un punto preciso dove gli interessa maggiormente essere informato delle condizioni meteo della zona.

\par I principali problemi che vengono attualmente riscontrati sono i seguenti:
\begin{itemize}
    \item \textbf{Diverse Fonti:} l'utente deve consultare più fonti, non fidandosi di un singolo parere, dato che generalmente non tutti i siti dedicati concordano sui risultati.
    \item \textbf{Affidabilità:} l'utente deve tenere a mente l'affidabilità dei singoli siti e/o calcolarne approssimativamente il peso sulla base delle proprie esperienze pregresse con gli stessi.
    \item \textbf{Customizzazione:} l'utente non ha possibilità nei siti più famosi di impostare una propria centralina meteo da accompagnare alle previsioni tradizionali.
\end{itemize}
 
\section{Obiettivi}
L'obiettivo consiste nel fornire all'utente un servizio per la consultazione delle previsioni meteo che risolva o quanto meno attenui i disagi o lo stress causato dai problemi sopra descritti. 

\par Nello specifico si vogliono incontrare le esigenze di chi ne fruisce, agevolando le operazioni di:
\begin{itemize}
    \item \textbf{Osservazione delle condizioni meteo}, sia attuali che fino a 5 giorni, per una determinata località.
    \item \textbf{Registrazione di un account}, permettendo il salvataggio di preferenze su una determinata località di cui ricevere le previsioni tutti i giorni sui canali di comunicazione preferiti
    \item \textbf{Assegnazione di feedbacks} ai vari provider meteo lasciando un voto sulla precisione e correttezza delle previsioni emanate. 
    \item \textbf{Registrazione di una centralina} che permetta di avere dati personalizzati per tutti gli utenti consultatori del sito, per affinare le previsioni emanate dagli altri servizi meteo o per ricevere previsioni accurate su una specifica zona.
\end{itemize}





