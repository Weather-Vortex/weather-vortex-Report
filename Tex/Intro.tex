% -*- root: ../main.tex -*-

% Esporre l'obiettivo del progetto dandone una visione complessiva. Devono essere illustrate le caratteristiche salienti del progetto; deve essere chiara la distinzione tra le tecnologie usate/assemblate durante lo svolgimento dell'elaborato e il contributo tecnologico/scientifico e effettivamente apportato dal gruppo.

\chapter{Introduzione}


\section{Overview}
Weather Vortex è un progetto che nasce con l'intento di fornire all'utente un unico servizio per consultare e paragonare diverse previsioni meteo. L'obbiettivo verrà raggiunto mediante la creazione di una \textbf{piattaforma web} che possa, da un lato, \textbf{facilitare le decisioni dell'utente}, dall'altro \textbf{fornire un'esperienza personalizzata}.

\begin{figure}[H]
    \caption{Il logo della piattaforma}
    \label{fig:Logo}
    \centering
    \includegraphics[width=0.5\textwidth]{Images/logo.png}
\end{figure}

\section{Problemi}
Normalmente un utente consulta in media 3 / 4 siti meteo ogni mese per farsi un'idea precisa delle previsioni. Questo comporta un dispendio di tempo e risorse nel consultare più fonti, quando un aggregatore potrebbe fare tutto in automatico.

I principali problemi che vengono attualmente riscontrati sono i seguenti:
\begin{itemize}
    \item \textbf{Diverse Fonti:} l'utente consulta più fonti, non fidandosi di un singolo parere.
    \item \textbf{Affidabilità:} l'utente deve tenere a mente l'affidabilità dei singoli siti e/o calcolarne approssimativamente il peso.
    \item \textbf{Customizzazione:} l'utente non ha possibilità nei siti più famosi di impostare una propria centralina meteo da accompagnare alle previsioni tradizionali.
\end{itemize}
 

\section{Obiettivi}
L'obiettivo consiste nel fornire all'utente un servizio omnicomprensivo per le previsioni meteo. 

Nello specifico si vogliono incontrare le esigenze di chi lo fruisce, agevolando le operazioni di:
\begin{itemize}
    \item \textbf{Osservazione delle previsioni} fino a 5 giorni
    \item \textbf{Registrazione di un account} che permette di salvare le preferenze e aggiungere feedbacks ai vari provider meteo
    \item \textbf{Registrazione di una centralina} che permetta di avere dati personalizzati
\end{itemize}





