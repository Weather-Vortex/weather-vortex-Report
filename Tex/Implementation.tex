% -*- root: ../main.tex -*-

% Esporre i principali problemi affrontati durante l'effettiva realizzazione delle componenti hardware/software e illustrare le soluzioni implementative adottate. Se l'elaborato ha previsto l'utilizzo di tecnologie già disponibili sul mercato, discuterne brevemente le caratteristiche e motivarne l'adozione rispetto ad altre soluzioni assimilabili. NOTA: in questa sezione devono essere riportate esclusivamente le porzioni di codice ritenute particolarmente significative.

\chapter{Implementazione}
\section{Server}
\textit{Commento: questo secondo me è da fare insieme, ognuno può aggiungere qualcosa della propria parte. Magari scrive uno e l'altro se vuol dire qualcosa anche della sua parte la aggiunge.}

Per la parte backend ci si è avvalsi della tecnologia Node.js, Express e MongoDb per il database. Per l'invio delle email è stato utilizzato un modulo di Node.js chiamato Nodemailer.


Per supportare la comunicazione bidirezionale tra client e server ci si è avvalsi della libreria Sockets.Io.


Si riportano ora alcuni aspetti implementativi del server che si ritiene opportuno evidenziare. 

\subsection{Forecast e Socket.IO}

All'avvio del Server, viene subito richiamata la funzione di Sockets.IO per rimanere in ascolto di nuove connessioni da parte delle sockets dei clients per richiedere le previsioni meteo:

\begin{lstlisting}[language=Javascript]
io.on("connection", (socket) => {
  socket.on("current", (arg) => {
    if (arg.locality) {
      // Retrieve forecasts by locality.
    } else if (arg.latitude && arg.longitude) {
      // Retrieve forecasts by geolocation position.
    });
    
  socket.on("threedays", (arg) => {
    // Same as before.
  });
});
\end{lstlisting}

In seguito, quando i risultati delle richieste di previsioni meteo ai servizi esterni sono pronti, vengono immediatamente inoltrati ai clients tramite gli eventi delle sockets:

\begin{lstlisting}[language=Javascript]
openWeatherStorage
  .currentByLocation(latitude, longitude)
  .then((result) => {
    socket.emit("result", ...);
  })
  .catch((error) => {
    socket.emit("forecast_error", ...);
  });
  
troposphereStorage
  .currentByLocation(latitude, longitude)
  .then((result) => {
    socket.emit("result", ...);
  })
  .catch((error) => {
    socket.emit("forecast_error", ... );
  });
\end{lstlisting}

\subsection{Autenticazione}

\subsubsection{JSON Web Token JWT}
Per l'autenticazione è stato deciso di utilizzare il sistema JWT (Json Web Token), firmati dal backend. Al momento del login il server rilascia un token al client contenente dei dati utili all’autenticazione. Il client, inviando al server il token ricevuto assieme alle richieste che necessitano di autenticare il richiedente, potrà essere riconosciuto e verificato dal server. In questa maniera il server rimane stateless, dato che le informazioni dell'autenticazione devono essere necessariamente inviate ad ogni richiesta da parte del client.

\begin{lstlisting}[language=Javascript]
// generate token when user log in
userSchema.methods.generateToken = function (cb) {
  var user = this;
  var token = jwt.sign({ _id: user._id }, process.env.SECRET);
  user.token = token;
  user.save(function (err, user) {
    if (err) return cb(err);
    cb(null, user);
  });
};

// find by token
userSchema.statics.findByToken = function (token, cb) {
  var user = this;
  jwt.verify(token, process.env.SECRET, function (err, decode) {
    user.findOne({ _id: decode, token: token },
      function (err, user) {
        if (err) return cb(err);
        cb(null, user);
      }
    );
  });
};

//delete token, when the user logout
userSchema.methods.deleteToken = function (token, cb) {
  var user = this;

  user.update({ $unset: { token: 1 } },
    function (err, user) {
      if (err) return cb(err);
      cb(null, user);
    }
  );
};

\end{lstlisting}

Per facilitare l'utilizzo della validazione del token JWT è stato prodotto un middleware facilmente inseribile nel flusso di Express. Questo middleware recupera il token disponibile dai cookies. Se lo trova associato ad un utente nel database, recupera i dati su di esso e li inserisce nella richiesta per renderli disponibili nell'handler della
route desiderata.

\begin{lstlisting}[language=Javascript]
let auth = (req, res, next) => {
  let token = req.cookies.auth;
  if (!token) {
    // Response the missing token result.
    return res.status(401).json({
      error: "Missing auth token in the request.",
    });
  }

  User.findByToken(token, (err, user) => {
    if (err) throw err;
    if (!user)
      // Notify the unauthorized access.
      return res.status(403).json({
        error: "Didn't found any user matching the auth token provided.",
      });

    // Use the found data in next handler.
    req.token = token;
    req.user = user;
    next();
  });
};
\end{lstlisting}

\subsubsection{Dati sensibili}

Si è avuta la necessità anche di gestire alcuni dati sensibili dell'utente, come le password. E' stato utilizzato bcrypt, un modulo contenente le più comuni funzioni di hashing per salvare le password in maniera sicura
nel database; viene utilizzato al momento della registrazione di un utente
quando si deve salvare la password (viene salvato il suo hash con il
relativo valore di sale), e al login quando bisogna confrontare la password
ricevuta dall’utente con quella salvata.

\begin{lstlisting}[language=Javascript]
userSchema.pre("save", function (next) {
  var user = this;
  if (user.isModified("password")) {
    bcrypt.genSalt(salt, function (err, salt) {
      if (err) return next(err);
      bcrypt.hash(user.password, salt, function (err, hash) {
        if (err) return next(err);
        user.password = hash;
        next();
      });
    });
  } else {
    next();
  }
});

userSchema.methods.comparePassword = function (password, cb) {
  // Compare the user password when user tries to login
  bcrypt.compare(password, this.password, function (err, isMatch) {
    if (err) return cb(next);
    cb(null, isMatch);
  });
};
\end{lstlisting}

\section{Client}
\textit{Commento:anche questo da completare come nel server, ci ho messo qualche spunto}\\
Per lo sviluppo del Client è stato usato Vuetify, un framework di componenti di Material Design per Vue. js che consente agli sviluppatori di creare incredibili applicazioni in modo rapido ed efficiente.
Si riportano ora alcuni aspetti implementativi del server che si ritiene oppurtuno evidenziare: 
\subsection{Socket}
TODO

\subsection{Store}
Per gestire lo stato dell'applicazione, abbiamo avuto la necessità di sfruttare il framework Vuex. Esso fornisce un'interfaccia unica per tutta l'applicazione per interrrogare lo stato e modificarlo. Questo permette di avere un'unica fonte di verità dei dati e ne preserva maggiormente l'integrità.

Graficamente potremo delinearne così la struttura:
\begin{figure}[H]
    \caption{Le azioni sono fuori dallo store e diventano servizi}
    \label{fig:Store}
    \centering
    \includegraphics[width=0.7\textwidth]{Images/store.png}
\end{figure}

Il nostro stato è rappresentato dal listato sottostante, in particolare abbiamo memorizzato la visibilità della barra di navigazione e lo stato di autenticazione dell'utente.

\begin{lstlisting}[language=Javascript]
const state = {
  drawer: null,
  user: null,
};
\end{lstlisting}

Lo stato viene quindi letto quando bisogna mostrare o no la barra di navigazione oppure capire quando un utente è autenticato oppure no.

\section{Centralina}
L'applicativo per la centralina è stato sviluppato anch'esso con le tecnologie Node.JS ed Express. Essendo un progetto di simulazione di una centralina reale, dispone di solamente due funzioni: una per ottenere info sullo stato del dispositivo e una per ottenere le condizioni meteo correnti (generate staticamente, data la natura del progetto).

\begin{lstlisting}[language=Javascript]
const express = require("express");
const weather = require("./storage/weather.storage");

const app = express();

app.get("/current", weather.readAll);

app.get("/info", (req, res, next) =>
  res.status(200).json({
    ...
  });
);
\end{lstlisting}

\section{Prodotto Finale}
\label{prodottofinale}
Il prodotto finale ha subito delle evoluzioni rispetto al mockup ma segue le linee
guida imposte.
TODO: Screen di esempio dell'applicazione