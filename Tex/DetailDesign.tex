% -*- root: ../main.tex -*-
\chapter{Design di Dettaglio}
\section{Target Users}
\section{Scenario d'uso}
\section{Frontend}
\subsection{A ogni sottosezione descrivere una pagina}
\section{Backend}
\subsection{Risorse}
Il backend è stato strutturato secondo le seguenti risorse:
\begin{itemize}
\item auth - risorsa per registrare, autenticare gli utenti, cancellarli, recuperarne le
informazioni o aggiornarle.
\item feedback - risorsa per gestire le recensioni degli utenti sui vari provider meteo.
\item location - risorsa per cercare di recuperare informazioni di localizzazione di una determinata città
\item station - risorsa per gestire le varie stazioni meteo

\item openweathermap - risorsa per gestire le richieste dall'api openweathermap e recuperare le previsioni meteo desiderate
\item troposphere - risorsa per gestire le richieste dall'api troposphere e recuperare le previsioni meteo desiderate
\item users - risorsa per recuperare i feedbacks e le stazioni di un determinato utente.
\item weatherProvider - risorsa per recuperare il meteo dalle stazioni
\end{itemize}
\subsection{Autenticazione}
E stato deciso di utilizzare il sistema di autenticazione a token JWT, firmati `
dal backend.
Nel token è inserito l'id dell'utente, quindi tutte le operazioni che lo
richiedono lo recuperano da esso.
Quando un utente esegue il login correttamente, il backend invia al frontend
il token da usare e quest’ultimo lo inserisce nel header x-access-token di tutte
le richieste che lo richiedono.
Questo flusso è facilmente integrabile nel flusso delle richieste e delle comunicazioni tra il backend e il frontend.

\section{Prodotto finale}
TODO: Screen di esempio dell'applicazione
