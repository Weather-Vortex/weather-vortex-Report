% -*- root: ../main.tex -*-

\chapter{Conclusioni}
    \section{Commenti Finali}
        \subsection{Daniele Tentoni}
        \subsection{Silvia Zandoli}
        \subsection{Igor Lirussi}
        Al termine del percorso di sviluppo del progetto ritengo di aver acquisito una discreta moltitudine di nuove conoscenze. Sono state impiegate e integrate diverse tecnologie e metodologie anche complesse con cui sono state affinate parecchio le mie competenze nell'ambito. Sicuramente c'è ancora spazio per approfondire molti argomenti, ma nel complesso reputo sia servito a formare una base di conoscenza a tutto tondo. Il lavoro di gruppo ha permesso, infatti, di scambiarci nozioni a vicenda e ne sono più che contento, seppur mi rincresce non essere stato presente a volte come avrei voluto. Molte tecnologie usate sono state per me una scoperte \emph{in-itinere}, in quanto completamente nuove. Ritengo comunque il progetto sia stato particolarmente ambizioso e il risultato ottenuto a mio parere è più che soddisfacente. Infine, spero questo lavoro possa rappresentare una buona base di partenza per un eventuale sviluppo futuro. 
    \section{Sviluppi Futuri}
    In futuro, il progetto potrà essere migliorato nelle sue parti lasciando spazio a nuove tecnologie e, auspicabilmente, anche ad una commercializzazione. 
    Le aggiunte che abbiamo individuato sono sia di natura grafica che funzionale, ma, a causa delle tempistiche ridotte e del carico di lavoro, sono state demandate ad un'implementazione futura. \newline
    Eventualmente la parte funzionale si può estendere creando  nuove pagine per l'utente finale, con sezioni quali "notizie", "video degli utenti", "allerte meteo". Inoltre nuove funzioni possono sempre essere aggiunte al back-end per migliorare la parte di aggregazione delle varie previsioni con machine learning, ponderamento in base ai feedback o allo storico delle condizioni meteo reali. Può essere implementata una parte amministrativa del sito e integrata in maniera non invasiva della pubblicità. \newline
    La parte grafica invece può essere ulteriormente migliorata con animazioni e elementi accattivanti, eventualmente che rispecchino le attuali condizioni meteo. Inoltre mappe interattive possono aiutare l'utente ad avere un'idea più chiara della situazione geografica. Eventualmente anche temi personalizzati possono essere sviluppati per rispecchiare i gusti degli utilizzatori.  \newline
    Negli sviluppi futuri includiamo anche un interfacciamento con i social maggiormente diffusi, oramai essenziali allo sviluppo di un business. In questo modo si potrebbe automatizzare la creazione di post, facilitare l'interazione tra gli utenti, lo sviluppo di una community e l'esposizione della piattaforma.  \newline
    Inoltre, come possibili spin-off sono state individuate delle aree tematiche spesso ignorate in cui le previsioni sono necessarie, ad esempio previsioni per lo sport, quali venti per i praticanti di parapendio o correnti marine per i surfisti. Anche qui è altamente necessario l'utilizzo di device IoT.  \newline
    A riguardo, in futuro si potrebbe pensare ad una commercializzazione di un device integrato "ready to use" per delle previsioni più personali.  \newline
    In conclusione in futuro il progetto offre numerose opportunità di ampliamento, con interessanti prospettive sia dal lato tecnologico che dal lato business.  
    
