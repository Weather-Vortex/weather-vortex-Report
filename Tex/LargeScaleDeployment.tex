% -*- root: ../main.tex -*-

% In questa sezione va discussa, eventualmente con l'ausilio di opportuni diagrammi (componenti, deployment), l'evoluzione del progetto presentato immaginando che venga adottato su larga scala. I dettagli qui esposti devono quindi astrarre dalle specifiche dell'elaborato qualora l'implementazione sia stata focalizzata su uno scenario isolato. A titolo d'esempio, qualora applicabile, devono essere evidenziate le criticità che si potrebbero incontrare e devono essere proposte soluzioni tipiche in contesti di cloud architecture per garantire un'adeguata resilienza, in termini di availability e scalability del sistema.

\chapter{Analisi di Deployment}
\section{Avvio dell'applicazione}
Per dispiegare tutti i componenti utili per l’applicazione è stato impiegato il
gestore di container Docker e la sua estensione Docker-compose.
Per ciascun componente è stato creato il file di istruzioni per creare l’immagine
e utilizzando docker-compose è stato possibile avviare tutti i container utilizzando un solo comando.
Basterà infatti un unico comando per avviare l'intera applicazione : \textbf{[sudo] docker-compose up}